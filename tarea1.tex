% ====== TAREA 1 DEPROBABILIDAD ======

\documentclass[12pt,a4paper]{report}
\usepackage[utf8x]{inputenc}
\usepackage{amsmath}
\usepackage{amsfonts}
\usepackage{amssymb}
\usepackage{graphicx}
\begin{document}
\begin{titlepage}
	\centering
	{\scshape\LARGE Universidad Autónoma de México \par}
	\vspace{1cm}
	{\scshape\Large Probabilidad I\par}
	\vspace{1.5cm}
	{\huge\bfseries Tarea I\par}
	\vspace{.5cm}
	{\Large\itshape Sandra Del Mar Soto Corderi \par}
	\vspace{.5cm}
	{\Large\itshape Edgar Quiroz Castañeda \par}
    \vspace{.5cm}
	{\Large\itshape Raúl Llamosas Alvarado \par}
	 \vspace{.5cm}
	{\Large\itshape Alan Ernesto Arteaga Vázquez \par}
	\vfill
	
	\vfill

% Bottom of the page
	{\large \today\par}
\end{titlepage}

\pagebreak
    1. Supongamos que existen 6 caminos para ir del edificio O al ’nuevo’
       edificio y 7 caminos para ir del ’nuevo’ edificio al Yelizcalli.
       ¿Cuántos caminos existen para ir del edificio O al Yelizcalli
       (pasando por el nuevo edificio)?\\


    2. Un estudiante tiene 10 amigos, de los cuales 6 serán invitados.\\

        a) ¿De cuántas formas pueden ser seleccionados los 6 amigos?

        b) ¿De cuántas formas pueden ser seleccionados si 2 de los amigos están
            peleados y no asisitirán juntos?\\\\

    3. ¿Cuántas palabras (con o sin sentido) son posibles de formar usando todas
       las letras de la palabra\\

       a) esternocleidomastoideo?

       b) polen?\\\\


    4. Sea $S = \{1, 2, 3, ... , n\}$ ¿Cuántos subconjuntos de S se pueden
       formar que tengan algún elemento en particular?. ¿Qué proporción
       representa del total de todos los posibles subconjuntos de $S$?\\\\

    5. ¿De cuántas formas se pueden sentar 8 personas en una fila si\\

        a) 5 personas son hombres y ellos se deben de sentar juntos?

        b) hay 4 parejas y cada pareja se debe sentar junta?\\\\

    6. Un estudiante de la Facultad de Ciencias tiene para escoger 7 preguntas
       de un total de 10 en un examen de probabilidad.\\

       a) ¿De cuántas formas puede hacer la elección de las preguntas?

       b) ¿De cuántas formas puede puede hacer la elección si debe de responder
          al menos tres preguntas de las primeras 5?\\\\

\end{document}
