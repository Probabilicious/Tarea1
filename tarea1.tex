% ====== TAREA 1 DE PROBABILIDAD ======

\documentclass[12pt,a4paper]{report}
\usepackage[utf8x]{inputenc}
\usepackage{amsmath}
\usepackage{amsfonts}
\usepackage{amssymb}
\usepackage{graphicx}
\usepackage{ragged2e}

\newcommand*{\Comb}[2]{{}^{#1}C_{#2}}

\begin{document}
\begin{titlepage}
	\centering
	{\scshape\LARGE Universidad Autónoma de México \par}
	\vspace{1cm}
	{\scshape\Large Probabilidad I\par}
	\vspace{1.5cm}
	{\huge\bfseries Tarea I\par}
	\vspace{.5cm}
	{\Large\itshape Sandra Del Mar Soto Corderi \par}
	\vspace{.5cm}
	{\Large\itshape Edgar Quiroz Castañeda \par}
    \vspace{.5cm}
	{\Large\itshape Raúl Llamosas Alvarado \par}
	 \vspace{.5cm}
	{\Large\itshape Alan Ernesto Arteaga Vázquez \par}
	\vfill

	\vfill

% Bottom of the page
	{\large Martes 21 de Agosto del 2018 \par}
\end{titlepage}

\pagebreak

\begin{enumerate}
   % Ejercicio 1
   \item {
   Supongamos que existen 6 caminos para ir del edificio O al ’nuevo’
   edificio y 7 caminos para ir del ’nuevo’ edificio al Yelizcalli.
   ¿Cuántos caminos existen para ir del edificio O al Yelizcalli
   (pasando por el nuevo edificio)?\\


	}

	% Ejercicio 2
   \item {
    Un estudiante tiene 10 amigos, de los cuales 6 serán invitados.\\
		\begin{enumerate}
		%a)
		\item{
		¿De cuántas formas pueden ser seleccionados los 6 amigos?\\

		$10 \choose 6 $ = $\Comb{10}{6}=\frac{10!}{6!(10-6)!}$ = 210 formas.\\
		}

		%b)
		\item{
        ¿De cuántas formas pueden ser seleccionados si 2 de los amigos están
        peleados y no asisitirán juntos?\\

        El estudiante puede invitar a uno de los dos amigos peleados o a ninguno \\

         $(\Comb{2}{1} \boldsymbol{\cdot} \Comb{8}{5}) +(\Comb{2}{0} \boldsymbol{\cdot} \Comb{8}{6})$ = 86 formas.\\
		}
	\end{enumerate}
	}


	% Ejercicio 3
   \item {
    ¿Cuántas palabras (con o sin sentido) son posibles de formar usando todas
    las letras de la palabra\\

	\begin{enumerate}
	% a)
   \item {
	esternocleidomastoideo?\\

   }

   % b)
   \item {
   polen?\\

   }
	\end{enumerate}

	}

	% Ejercicio 4
   \item {
    Sea $S = \{1, 2, 3, ... , n\}$ ¿Cuántos subconjuntos de S se pueden
    formar que tengan algún elemento en particular?. ¿Qué proporción
    representa del total de todos los posibles subconjuntos de $S$?\\\\

		Digamos que queremos formar subconjuntos de $S$ que contengan a un elemento
		$s \in S$.
		Hay $2^n$ subconjunto de S. Además, todos los subconjuntos de $S$ que no
		contengan a $s$ son también subconjuntos de $S / {s}$, que tiene $2^(n-1)$
		posibles subconjuntos. Entonces la cantidad de subconjuntos de $S$ que
		contienen a $s$ son $2^n - 2^(n-1) = 2^n - 2^n * 2^-1 = 2^n(1 - 2^-1) =
		2^n(2^-1) = 2^(n-1)$
		Y la proporción de subconjuntos que lo cumplen son $\frac{2^⁽n-1), 2^n} =
		 \frac{2^n * 2^-1, 2^n} = 2^-1 = \frac{1, 2}$\\
	}

	% Ejercicio 5
   \item {
    ¿De cuántas formas se pueden sentar 8 personas en una fila si\\

	\begin{enumerate}
	% a
   \item {
   ¿5 personas son hombres y ellos se deben de sentar juntos?\\

   }

   % b
   \item {
   ¿Hay 4 parejas y cada pareja se debe sentar junta?\\

   }

	\end{enumerate}

    }

	% Ejercicio 6
   \item {
    Un estudiante de la Facultad de Ciencias tiene para escoger 7 preguntas
    de un total de 10 en un examen de probabilidad.\\

	\begin{enumerate}
	% a
   \item {
   ¿De cuántas formas puede hacer la elección de las preguntas?\\

   }

   % b
   \item {
   ¿De cuántas formas puede puede hacer la elección si debe de responder
   al menos tres preguntas de las primeras 5?\\

   }
	\end{enumerate}
    }
\end{enumerate}
\end{document}
